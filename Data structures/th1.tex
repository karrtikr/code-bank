%%% Template originaly created by Karol Kozioł (mail@karol-koziol.net) and modified for ShareLaTeX use

\documentclass[a4paper,11pt]{article}

\usepackage[T1]{fontenc}
\usepackage[utf8]{inputenc}
\usepackage{graphicx}
\usepackage{xcolor}

\renewcommand\familydefault{\sfdefault}
\usepackage{tgheros}
\usepackage[defaultmono]{droidmono}

\usepackage{amsmath,amssymb,amsthm,textcomp}
\usepackage{enumerate}
\usepackage{multicol}
\usepackage{tikz}

\usepackage{geometry}
\geometry{
left=25mm,right=25mm,%
bindingoffset=0mm, top=20mm,bottom=20mm}

\def\code#1{\texttt{#1}}
\linespread{1.3}

\newcommand{\linia}{\rule{\linewidth}{0.5pt}}

% custom theorems if needed
\newtheoremstyle{mytheor}
    {1ex}{1ex}{\normalfont}{0pt}{\scshape}{.}{1ex}
    {{\thmname{#1 }}{\thmnumber{#2}}{\thmnote{ (#3)}}}

\theoremstyle{mytheor}
\newtheorem{defi}{Definition}

% my own titles
\makeatletter
\renewcommand{\maketitle}{
\begin{center}
\vspace{2ex}
{\huge \textsc{\@title}}
\vspace{1ex}
\\

\vspace{4ex}
\end{center}
}
\makeatother
%%%

% custom footers and headers
\usepackage{fancyhdr}
\pagestyle{fancy}
\lhead{}
\chead{}
\rhead{}
\lfoot{Assignment \textnumero{} 5}
\cfoot{}
\rfoot{Page \thepage}
\renewcommand{\headrulewidth}{0pt}
\renewcommand{\footrulewidth}{0pt}
%

% code listing settings
\usepackage{listings}
\lstset{
    language=C,
    basicstyle=\ttfamily\small,
    aboveskip={1.0\baselineskip},
    belowskip={1.0\baselineskip},
    columns=fixed,
    extendedchars=true,
    breaklines=true,
    tabsize=4,
    prebreak=\raisebox{0ex}[0ex][0ex]{\ensuremath{\hookleftarrow}},
    frame=lines,
    showtabs=false,
    showspaces=false,
    showstringspaces=false,
    keywordstyle=\color[rgb]{0.627,0.126,0.941},
    commentstyle=\color[rgb]{0.133,0.545,0.133},
    stringstyle=\color[rgb]{01,0,0},
    numbers=left,
    numberstyle=\small,
    stepnumber=1,
    numbersep=10pt,
    captionpos=t,
    escapeinside={\%*}{*)}
}

%%%----------%%%----------%%%----------%%%----------%%%

\begin{document}

\title{Theoritical Assignment 1}

\begin{flushright}
Kartik Raj - 14308\\
Sanjay Ramesh - 14601
\end{flushright}
\maketitle

\section*{Maximum Sum Submatrix}

We are given the matrix A as input which is nxn.
\subsection*{Data Structure}
1. nxnxn matrix S where S(i,j,k) denotes the maximum sum submatrix starting at kth row and ending at index (i,j).\\ O($n^3$) space\\
2. nxn matrix s where s(i,j) denote the maximum sum submatrix ending at index (i,j).\\ O($n^2$) space\\
3. nxnxn matrix B where B(k,i,j) denote the sum from kth row to ith row in jth column.\\ O($n^3$) space\\
Space complexity of Algorithm = O($n^3$)
\begin{lstlisting}[label={list:first},caption=Algorithm to calculate Maximum Sum Submatrix]
Maxsum-submatrix(A)
{
	"Data Structure Involved"
	for i=0 to n-1{
		for k=0 to i{
            S(i,0,k)="Summation of a[x][0] over x in range k to i"
    	}	//O(n cube) time complexity
	}
    "Initialize each element of B(k,i,j) to be 0"
    for j=0 to n-1{
    	for i=0 to n-1{
        	for x=0 to i{
            B(0,i,j)=B(0,i,j)+a[x][j];
            }	//Initialized B(0,i,j) as per definition
        }	//O(n cube) time complexity
    }
    for j=0 to n-1{	//Filling the B(k,i,j) matrix
    	for i=1 to n-1{
        	for k=1 to i{
            	B(k,i,j)=B(k-1,i,j)-a[k][j];
            }	//O(n cube) time complexity
        }
    }	//Note k<=i are the only valid entries as per definition
    "Algorithmic part"
    for i=0 to n-1{
    	for j=1 to n-1{
        	for k=0 to i{
            	if S(i,j-1,k)>0 then S(i,j,k)=S(i,j-1,k)+b(k,i,j);
                else S(i,j,k)=b(k,i,j);
            }
            s(i,j)="Max of S(i,j,k) over k in range 1 to i"
        }	//O(n cube) time complexity
    }
    "Scan s to return the maximum entry"	//O(n square) time complexity
}
\end{lstlisting}

Following Listing~\ref{list:first}\ldots{} 
Lorem ipsum dolor sit amet, consectetur adipiscing elit, sed do eiusmod tempor incididunt ut labore et dolore magna aliqua. Ut enim ad
minim veniam, quis nostrud exercitation ullamco laboris nisi ut aliquip ex ea commodo consequat. Duis aute irure dolor in reprehenderit in 
\\
\section*{Range-Minima Problem-2nd part}

We are given 1-D order n array A as input.
\subsection*{Data Structure}
1. Array B where B[i] stores minimum of A[i] and A[i+1]\\ O($n$) space\\
2. Array Power-of-2[ ]\\ O($n$) space\\
3. Array Log[ ]\\ O($log(n)$) space\\
4. Array Two-Power[ ] where ith index stores the value of 2 to the power i\\ O($n$) space\\
Space complexity of Algorithm = O($n$)

Let Min be a function where Min(i,k) returns the minimum of \{$A[i],A[i+1],A[i+2],..,A[i+2^k]$\}
\begin{lstlisting}[label={list:second},caption=Range-Minima Algorithm]
Range-Minima(B,i,j){
	L=j-i;
    t=Power-of-2[L];
    k=Log[L];
    if(t=L) return Min(B,i,k);
    else 	return "Minimum of Min(B,i,k) and Min(B,j-t,k)";}
Min(B,i,k){
	if k=0 return B[i];
    else{
    	return "Minimum of {Min(B,i,k/2),Min(B,i+Two-Power(k/2),k/2),A[i+Two-Power(k)]}"
    }
}
\end{lstlisting}

Lorem ipsum dolor sit amet, consectetur adipiscing elit, sed do eiusmod tempor incididunt ut labore et dolore magna aliqua. Ut enim ad minim veniam, quis nostrud exercitation ullamco laboris nisi ut aliquip ex ea commodo consequat. Duis aute irure dolor in reprehenderit in voluptate velit esse cillum dolore eu fugiat nulla pariatur. Excepteur sint occaecat cupidatat non proident, sunt in culpa qui officia deserunt mollit anim id est laborum.

\end{document}

